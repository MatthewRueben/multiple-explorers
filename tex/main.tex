% Paper for Learning-Based Controls (ME 537)

\documentclass[letterpaper, 10 pt, conference]{ieeeconf}  % Comment this line out if you need a4paper

\IEEEoverridecommandlockouts                              % This command is only needed if 
                                                          % you want to use the \thanks command

\overrideIEEEmargins                                      % Needed to meet printer requirements.


%\usepackage{ijcai09}  % style
\usepackage{times}    % font
\usepackage{graphicx} % inserting images
\usepackage{cite}
\usepackage{mathtools} % For math
\usepackage{hyperref}
%\usepackage{enumitem}
\renewcommand{\deg}{\ensuremath{^{\circ}}\xspace}  % why doesn't this work???

\providecommand{\e}[1]{\ensuremath{\times 10^{#1}}}

\graphicspath{ {./figures/} } % Point to the figures directory

%%%%%%%%%%%%%%%%%%%%%%%%%%%%%%%%%%%%%%%%%%%%%%%%%%%%%%%%%%%%%%%%%%%%%%%%%%%

\title{\LARGE \bf 
O Rover Where Art Thou?\\
Multi-Robot Exploration with Limited Sensor Field-of-View
}

\author{Kory Kraft$^{1}$, Austin Nicolai$^{2}$, and Matthew Rueben$^{3}$% <-this % stops a space
%\thanks{*This work was supported by... }% <-this % stops a space
\thanks{*All authors are with the School of Mechanical, Industrial, and Manufacturing Engineering,
         Oregon State University,
         Corvallis, OR, USA. }% <-this % stops a space
\thanks{$^{1}$Corresponding author: {\tt\small kraftko@onid.oregonstate.edu}}%
\thanks{$^{2}$Corresponding author: {\tt\small nicolaia@onid.oregonstate.edu}}%
\thanks{$^{3}$Corresponding author: {\tt\small ruebenm@onid.oregonstate.edu}}%
}

\begin{document}

\maketitle
\thispagestyle{empty}
\pagestyle{empty}

\begin{abstract}
Contribution!
\end{abstract}


\section{Introduction}
The decreasing costs and increasing capabilities of computer hardware and software make
robots an increasingly attractive alternative to human labor. More specifically, robots have 
been targeted for use in the exploration domain, both in single agent and 
multiagent systems.  Robots have the potential to
handle certain extreme environments better than their human
counterparts.  They can cover large swaths of territory more
efficiently, or conversely fit in very tight spaces without disturbing
the surrounding environment (e.g. snake robots). 

A number of inroads have been made into harnessing the power and
advantages of robots in the exploration domain. NASA continues to use
the rover Curiosity to explore Mars. Several companies have built
mobile robots to explore and decontaminate the the Fukushima Daiichi
nuclear power station after the 2011 meltdown CITE THE URL. Added to these
accomplishments, robots are increasingly being used with other robots, 
adding significance to multiagent systems (MAS) robot studies.

Previous MAS research however, often fails to fully consider the effect of
sensor limitations upon the learning algorithm with given domains. In many realistic cases, sensors fall drastically short of being able to see the entire search space at once. Many lower-cost robots are now using the Microsoft Kinect because it offers RGB-D sensing for fewer than 100 dollars. Albeit, the Kinect is constrained by a 0.8 - 4.0m distance range and 57$^{\circ}$ horizontal and 43$^{\circ}$ vertical field of view. On top of all this, there is ever present sensor noise; a constant source of frustration for sensor precision and accuracy.

As always, decreasing hardware costs does not mean infinite hardware resources.
Implementers must be able to informatively decide on the cost-benefit 
between a greater number of worse performing sensors and a greater array of lower performing sensors. Moreover, cost-benefit analsis needs to be done on a level that includes sensor impact on  agent performance within a given domain. Specifically, we will show the trade-offs between using varying quality sensors and the overall system performance by modeling robots with different sensor ranges, fields of view, and noise characteristics in the rover domain utilizing three common reward structures. This will help future implementers conduct more accurate cost-benefit analyses in order to make wiser decisions when building and deploying robots in the real world.


\section{Related Work}
\subsection{Credit Assignment}
A second unique aspect of a multi-agent system is the problem of credit assignment. Agent to agent coordination involves individual agents working together on a common task or goal.  Since one agent’s behaviors might affect other agents, however, the learning process becomes hard. Chalkiadakis and Boutilier \cite{chalkiadakis2003coordination} develop a way to weigh the costs and benefits of taking non-greedy actions during
reinforcement learning based on the value of the new information that might be gained. They also give the agents some limited ability to reason about how their actions influence others. Another question is, what does each agent know about the other agents? This is the
information by which they can coordinate at all. Gmytrasiewicz and Durfee \cite{gmytrasiewicz2000rational} expand the typical state space of each agent by adding knowledge about other agents’ knowledge, which is recursive by nature. They develop a framework whereby this representation is usable, however, and present their findings as a move towards coordinated robot actions. Finally, Burgard et al. \cite{burgard2005coordinated} address coordination in an exploration scenario, which for them boils down to intelligently placing goal locations (or waypoints, by extension) for
each agent.

\subsection{Exploration Domain}
In the exploration domain, agents must discover all areas of an initially unknown map. One method tackles the search of complicated areas via cellular decomposition and systematic coverage \cite{choset2001coverage}. Another method examines the effect of limited knowledge on an agent. It examines how well different algorithms utilize agents, as well as how the number of rooms and room size effects the explorability of a space \cite{ferranti2007brick}. Lastly, one method coordinates exhaustive exploration of a space by intelligently assigning destinations to different agents \cite{burgard2005coordinated}.

\subsection{Sensor Limitations}
A limited sensor range causes two potential problems for agents:
collision with each other and sub-optimal behavior due to lack of
coordination. Dimarogonas et al. \cite{dimarogonas2007decentralized} and Do \cite{do2007bounded} present planning approaches to collision avoidance with limited sensor range. To plan agent motion, both approaches use potential fields (also called
navigation functions) that include some potential or proximity function between each pair of agents that can see each other. The ant colony problem presented by Gordon et al. \cite{gordon2004gathering} examines the case where the sensors have limited range and no distance information, i.e., only output the direction to sensed entities. All of these approaches involve sending the agents to some goal location or formation, not exploring an area, and no machine learning techniques are used. One approach that used reinforcement learning examined reward-shaping in a noisy domain involving competitive robotic “foraging”. It was shown that intelligent reward-shaping and algorithm were able to overcome the noise present and achieve good performance\cite{mataric1997reinforcement}.

\subsection{Noise}
**TODO update citations**
Noisy sensors are a reality for robots. Lower-cost sensors, like the Microsoft Kinect, have the potential to significantly drive down costs for using robots for exploration. However, lower-cost sensors also bring with them higher noise characteristics. Specifically the Microsoft Kinect has been shown to have a random error in the depth axis ranging from a few millimeters to 4 cm in a non-uniform distribution \cite{blah} [1][2]. Techniques have been developed to overcome the various types of sensor noise, both by modifying sensor output and modifying learning techniques, but these still leave sensor information imperfect \cite{blah} [2][3]. Before spending more time and resources trying to overcome the sensors with added or improved hardware or software,it is important to understand how sensor noise actually impacts rover performance within a given domain.


\section{Methods}
\subsection{Domain}
This work is a direct extension of that described by Tumer and Agogino
(2008, \cite{agogino2008analyzing}). Therein, the goal of the rovers is to observe the more
important points of interest (POIs) at close range in order to earn a
large reward. We use their dynamic domain rather than their static
domain, so the POI values and locations are chosen randomly at the
beginning of each learning episode. For each learning episode, the
rovers begin in a tight cluster at the center of the map. The episode
ends after 15 time steps and rewards are assigned. Each rover can move
continuously in two dimensions, and 10 percent noise is added to the
intended movement to simulate real-world errors.

\subsection{Reward Structure}
The global reward presented by Tumer and Agogino (2008) is based on
the closest observation of each POI by any rover:
  
\[
G= \sum_j  \frac{V_j}{min_i \: \delta(L_j, L_i)}
\]

The distance function is the square of the Euclidean norm up to a
minimum distance d2:

\[
\delta  (x,y) = max \big\{||x - y||^2, d^2 \big\}    
\]

The difference reward would be more learnable for an individual rover,
but requires each rover to calculate the hypothetical global reward if
that rover were removed from the system. For this work, we use the
partial observation difference reward, which only accounts for the
rover in question and any rovers within its sensor range. If a rover
is the closest rover that it knows of to a given POI, it is credited
with the reward for that POI:

\[
D_i(PO) = \sum_j  \frac{V_j}{min_{i' \in O^\rho_i} \: \delta(L_j, L_{i'})} - \sum_j  \frac{V_j}{min_{i' \in O^\rho_i,  \neq i} \: \delta(L_j, L_{i'})}
\]

The authors showed this to be a highly factored, learnable, and
practical credit assignment structure.

\subsection{State Information}
Each rover has two types of sensors. The first type returns an
aggregate score that represents how many high-value POIs are near the
rover:

\[
s_{1,q,i} =  \sum_{J \in I_q}  \frac{V_j}{\delta (L_j, L_i)} 
\]

The second sensor type returns a similar aggregate for other rovers:

\[
s_{2,q,i} =  \sum_{i' \in N_q}  \frac{V_j}{\delta (L_{i'}, L_i)}  
\]

Each rover’s surroundings are divided into four quadrants and two
sensors -- one POI sensor and one rover sensor -- look out on each
quadrant. Eight total sensors therefore yield eight continuous state
variables.

\subsection{Action Selection and Learning}
State information was mapped to action choices using a perceptron with
8 inputs, 10 hidden nodes, and 2 outputs. The activation functions
were sigmoidal and each output value was scaled based on a maximum
movement allowance per step (d is maximum movement per time step and
o1, o2 are the perceptron outputs):

\[
dx = 2d(o_1-0.5)
\]
\[
dy = 2d(o_2-0.5)
\]

An evolutionary algorithm was used with simulated annealing to find
neural networks that yield good performance. This is called “direct
policy search” because we use an evolutionary algorithm to search for
a good policy directly, i.e., without maintaining state-action value
estimates as in approaches like Q-learning. The system described thus
far provides the basis for our study of sensor limitations. Our novel
study regarding sensor impact is detailed below.

\subsection{Sensor Evaluation}
Sensor impact upon learning was evaluated by comparing the baseline
system performance from \cite{agogino2008analyzing} with the system performance after limiting
the sensors in three ways: reducing range, adding noise, and reducing
the field-of-view angle.  Following \cite{agogino2008analyzing}, we defined system performance
as the sum global reward achieved.  Noise was defined as the random
error in a rover’s distance measurements. All three types of sensor
limitations were applied to the POI sensors and the rover sensors
equally. Due to the 2D nature of the simulated world, height was
ignored and sensor field-of-view angle was only measured about the
vertical axis.

Each sensor limitation was applied independently at three different
magnitudes.  Thus, we had 9 test cases total plus the 1 baseline
case. The baseline sensor combination was chosen after \cite{agogino2008analyzing} as having
unlimited sensor range, zero noise, and a 360° field of view. Sensor
range was tested at 1 world unit, 3 world units, and unlimited world
units. Sensor noise was modeled as a unique normal distribution for
each sensed object with standard deviations at 0\%, 10\%, and 30\% of the
object’s true range. Sensor viewing angles were tested at 90$^{\circ}$, 270$^{\circ}$,
and 360$^{\circ}$.

100 statistical system runs were executed for 1000 episodes for each
sensor combination at three separate scales: 10, 50, and 100
homogeneous agents.  System performance was then plotted for each
sensor attribute at each scale in comparison with the baseline sensor
performance.


\section{Interactions between Domain and Learned Policy}
\subsection{Controller Sensitivity to State Information}
Very early in our implementation efforts, we saw that the learned policies of the rovers could be described in terms of two aspects:
\begin{description}
\item[\emph{Sensitivity or Reactivity}] \hfill \\
The way the rover learns to respond to changes in sensor (state) information with changes in direction (i.e., action). If it learns to do this, then we say that the rover is \emph{sensitive} to changes in its state information, or that it \emph{reacts} strongly to such changes. This behavior is most directly influenced by the neural network weights that mediate the connections between inputs and the single hidden layer. These input weights determine for each input (here, POI and rover sensors) how much influence, if any, it will have on the action output of the neuro-controller. 
\item[\emph{Direction Bias}] \hfill \\
If these input weights make the inputs too large or too small, they will leave the range within which the sigmoidal activation functions of the hidden nodes are sensitive (i.e., about -5 to 5). In this case, changes in the input values (i.e., POI and rover sensor information) will cause only negligible changes to the hidden layer outputs, and the weights between the hidden layer and the output nodes will control the system. This causes behavior that is learned, but insensitive to sensor information; for our problem setup, this means a straight-line trajectory. The rover can only here learn the angle and speed of that trajectory. Assuming the speed is sufficient to reach the border of the map, the angle becomes very important for ensuring that the rovers spread out to cover all the POIs. 
\end{description}

Sensitivity can be estimated by how much an agent changes direction throughout its trajectory, especially as large sensor changes occur (e.g., passing by a rover or high-valued POI). A comparison of learned rover behavior that is sensitive and insensitive is shown in Figure \ref{fig:sensitivity}. Both systems have learned and indeed converged on their demonstrated behavior. 

\begin{figure}[h!]
    \centering
    \includegraphics[width=0.3\textwidth]{ATRIAS.jpg}
    \caption{Visualization of learned rover behavior over one episode. The reward was DIFFERENCE/GLOBAL/LOCAL. LEFT: rovers learned straight-line paths that ignore changing sensor information. This is sub-optimal (IS IT??? Doesn't matter to me, just say which it is for whatever we plot) for this domain. RIGHT: a different domain (DESCRIBE IT!) encouraged learned sensitivity to changing sensor information. This is apparent because the rovers change directions as the episode progresses. The direction changes are effective at observing more POIs more closely. }
    \label{fig:sensitivity}
\end{figure}


\subsection{Easy and Hard Domains}
We will now discuss the extent to which a domain can encourage or discourage learned sensitivity. 

The first consideration is whether sensor information is even necessary for earning the optimal system reward. In other words, can the rovers fully observe every POI by taking well-distributed, straight-line paths? We can calculate this. The critical parameter is the angular separation between paths; if there are enough rovers and a small enough POI region, the space between paths will be so small that no POI could possibly \emph{not} be fully observed (Figure \ref{fig:triviality}). For our domain, the POIs are bounded within a 70x70 square region with a minimum observation distance of 5; for these parameters, 22 or more rovers could obtain the maximum system reward with straight-line paths. This explains why 30 agents caused us problems -- SPECIFICS.

This raises the question of what domain characteristics most encourage the rovers to learn to use their sensors. CHECK THIS AGAINST REAL RESULTS. HERE ARE MY GUESSES:
\begin{itemize}
\item Non-Trivial Domain, as defined above. If evenly-spaced, straight-line paths can't optimally observe all the POIs, then rovers will need to learn to serpentine to cover more space or loop back for another pass after an initial charge through the POI region. Since our setup does not include time as an input, serpentine or zigzag behavior is impossible without changing outputs from the POI and rover sensors. A clearer example of this is when the rover learns to turn around if it overshoots the POI region; this requires sensitivity to a sensor reading that means zero POIs are ahead and many POIs are behind. 
\item Varying Sensor Signals (Figure \ref{fig:signal-contrast}). The rovers need to experience a wide variety of sensor signals in order to learn how to respond to them. It's possible for a domain to be non-trivial as defined above but still present a rover with unchanging POI sensor signals (rover sensors will depend on agent policies, not the domain, at least directly). For example, a domain with very low minimum POI observation distances might require lots of maneuvering to optimally explore, but a very uniform distribution of these POIs would make sensor signals relatively uniform as well. Placing POIs in clusters or varying their relative values would make the POI sensor signals more sensitive to where the rover is located and to which direction it's facing. 
\item Attending to Sensors is well-correlated with Improved Reward. For learning to occur, these changing sensor signals must be usable by the rover to increase its assigned rewards. This has been done for us with apparent success by the sensor characteristics prescribed in \cite{agogino2008analyzing}.


%\item Moving POIs. If the rovers must perform well for a wide variety of POI positions and not just a static layout, they must vary their actions between episodes, or perhaps even between time steps.  This requires attention to sensor information in order to differentiate between different POI positions. Caveat: in a trivial domain, e.g., with too many rovers for the area to be covered, it doesn't matter how much the POIs move -- sensor information still isn't useful. Here is an example of POI movement that helps teach the rovers to react to sensor information: a single POI is randomly placed at one of the four corners of the map. 
\end{itemize}

\begin{figure}[h!]
    \centering
    \includegraphics[width=0.3\textwidth]{ATRIAS.jpg}
    \caption{Analysis of a trivial domain for our problem. Rovers are assumed to learn evenly-spaced, straight-line paths. Relevant parameters are the number of rovers, the size of the region containing POIs, and the minimum observation distance for the POIs. }
    \label{fig:triviality}
\end{figure}

\begin{figure}[h!]
    \centering
    \includegraphics[width=0.3\textwidth]{ATRIAS.jpg}
    \caption{Examples of POI distributions with different signal-to-noise ratios in terms of the POI sensors. LEFT: a uniform POI distribution yields a low SNR. A weighted-value (MIDDLE) or clustered (RIGHT) POI distribution yields a high SNR. }
    \label{fig:signal-contrast}
\end{figure}

\section{Future Work}
The analysis of both our domain and results presents several avenues for further research.

\subsection{Additional Domain Complexity}
The difficulties presented by our original, trivial, domain yield interesting possibilities for future work. Our analysis suggests that specific domain characteristics may have a large impact in system performance. Engineering more complex domains may allow us to observe interesting trade-offs between sensor characteristics and the system reward. Some possible domain changes may include movement restrictions and vision occlusion (e.g. impassable terrain or a wall). Another possibility would be allowing agents to temporarily increase their sensor capabilities, similar to obtaining a better vantage point from a hill. There may exist domains in which limiting one sensor characteristic impacts performance less than limiting another (e.g. limiting sensor range over FOV).

Additionally, changing the reward structure allows us to change the nature of our domain. Presenting the agents with more complex reward structures can influence their cooperation and learned policies. With more complex behavior required of the agents, the performance trade-offs between limited sensor characteristics may become more evident. One possible way to increase the reward structure complexity is to require simultaneous POI observation in order to obtain a reward. Another possibility is to explore the issue of temporal credit assignment. If agents are able to better understand which of their actions most contributed to their reward, they are better able to adjust their policy.

\subsection{Sensor Restriction Combinations}
An additional avenue for future work involves exploring additional sensor characteristic combinations. While our current work only dealt with varying one sensor characteristic at a time, this would more closely model real world sensors. In reality, sensors have physical limitations on all characteristics. This work could allow for performance vs. cost comparisons (e.g. perhaps a 10 dollar sensor with 90$^{\circ}$ FOV and 5 meter sensing range performs only 5 percent worse than a 500 dollar sensor with 90$^{\circ}$ FOV and 30 meter sensing range).


\subsection{Heterogeneous Agents}
Exploring the performance of different sensor characteristic combinations naturally leads to the question of how a multi-agent system performs when agents in the system have different sensing capabilities. This is an interesting problem because it allows for the simulation of cost constrained systems (e.g. should agents be equipped with several cheaper, specialized sensors or less powerful, but equivalent sensors) as well as analyzing the impact of agent failure (e.g. sensor malfunctions restricting FOV or actuator failures preventing agent movement). The presence of heterogeneous agents leads to many interesting questions including: Can equivalent system performance be obtained if some agents have diminished capabilities? If so, at what point does performance begin to drastically diminish? Will agents with different capabilities learn to perform tasks more suited to them (e.g. agents with long range sensors learning to bypass nearby POIs in favor of more distant POIs)?


\section{Results}

\begin{figure}[h!]
    \centering
    \includegraphics[width=0.45\textwidth]{SF_LocalReward.png}
    \caption{}
    \label{fig:}
\end{figure}

\begin{figure}[h!]
    \centering
    \includegraphics[width=0.45\textwidth]{SN_LocalReward.png}
    \caption{ }
    \label{fig:}
\end{figure}

\begin{figure}[h!]
    \centering
    \includegraphics[width=0.45\textwidth]{SR_LocalReward.png}
    \caption{}
    \label{fig:}
\end{figure}



\bibliographystyle{IEEEtran}
\bibliography{main}


\end{document}
